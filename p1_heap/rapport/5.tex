\section*{Testing of implementations}
\subsection*{Test Setup}
  We test by generating the graph in memory, and then run Dijkstas algorithm as described in Fredman and Tarjan \cite{fibheap}. We measure the time it take for Dijkstras Algorithm to finish with each of the three different priority queues. The test machine was a 2x quad-core Intel Xeon 2.8 GHz with 8 GiB 800MHz DDR2 ram. The program was build using gcc version 4.2.1 (Based on Apple Inc. build 5658) (LLVM build 2335.15.00).

  The program can be build by entering the folder \textbf{p1\_heap} and run the following command: \textit{make -f p1\_heap.mk all}. This requires MAKE and GCC to be installed. Then the compiled binary will be in the subfolder \textbf{Debug}, and can be run with \textit{./p1\_heap}.

\subsection*{Test Data}
  \subsubsection*{Graphs with many \textit{decrease key} operations} 
    Constructing a graph with many \textit{decrease key} operations is fairly easy. The problem is making sure the \textit{decrease key} operation force the Binary Heap to move the element. If the element after the \textit{decrease key} doesn't violate the Binary Heap property, it does not need to be moved and will be a constant operation, and thus it will match the Fibonacci Heap.

    The graph is generated with the following code:
    \begin{algorithm}
      \caption{\code{Generate graph} max decrease calls}
      \begin{algorithmic}[1]
	\REQUIRE number of vertices $n$
	  \FOR {$j$ := $1$; $j$ < $n$; $j$ := $j + 1$}
	    \STATE $weights[0][j]$ := $n*n -j$
	  \ENDFOR
	  \STATE $weights[0][n]$ := $1$
	  \FOR {$i$ := $n-1$; $i > 1$; $i$ := $i-1$}
	    \FOR {$j$ := $1$; $j < i$; $j$ := $j+1$}
	      \STATE $weights[i][j]$ := $n^2 - (n - i)n - j + 1$
	    \ENDFOR
	    \STATE $weights[i][i-1]$ := 1;
	  \ENDFOR
	\RETURN $weights$
\end{algorithmic}
\end{algorithm}

This graph will call \textit{decrease key} on all edges that is outgoing from the node return by \textit{delete min}, and the node will have an edge to all remaining nodes in the graph. Further more, all \textit{decrease key} calls will force the Binary Heap to bubble the element from a leaf to the root for every call, and thus maximizing the work done.
 \subsubsection*{Random Graphs}
