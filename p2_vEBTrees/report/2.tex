\section*{van Emde Boas Trees}
The van Emde Boas Tree data structure is recursive, this means we have to decide when to end the recursion. A natural starting point is to let the recursion end with leafs consisting of the 2 elements, $min$ and $max$, but since a tree with 3 elements will have a $TOP$ and $BOTTOM$ with a single element, that must also be supported. However, as stated in the note from G. Frandsen, and at the lectures, there is a point where you will gain better performance by reverting to an array and a liniar scan. Our implementation allows for this $threshold$ to be set upon creation of the tree. Given that we use 24 bits for the tree, and that the bit length is halfed every level in the tree, it should only have a performance impact if we change the depth of the tree. The cutoff points are at bit length 12, 6, and 3, i.e. leaf size of 4096, 64, and 8 respectively. Indeed quick testing show this is where we see a chance in the running time. Based on this quick testing we choose the threshold to be 64. Interestingly enough, if the tree was dense, it would also be an improvement to set the leaf size to 4096 - but this does not hold true for sparse trees.

\subsection*{Supported operations}
Our implementation support the following queries on a vEB tree:

\begin{itemize}
\item $uint32\_t \textbf{ veb\_insert}(uint32\_t index, void * data, vebtree * tree)$
\item $void \textbf{ veb\_delete}(uint32\_t index, vebtree * tree)$
\item $int32\_t \textbf{ veb\_findsucc}(uint32\_t index, void * data, vebtree *tree)$
\item $int32\_t \textbf{ veb\_findpred}(uint32\_t index, void * data, vebtree *tree)$
\end{itemize}

\textbf{delete\_min} and \textbf{find\_min} can both be implementeted with the above operations, by first calling \textbf{veb\_findsucc} on $0$, and if you want to delete it, call \textbf{veb\_delete} on the returned index.


\subsection*{Handling the leafs}
Our leafs are a simple array of elements and as such will not be explained in more detail than this part. The operations to \textbf{insert} and \textbf{delete} are a matter of inserting and deleting in the correct place in the array, this is done in constant time. The operations \textbf{find\_pred} and \textbf{find\_succ} are a matter of scanning the array from a certain posistion to find the first, if any, element that is before or after. The $min$ and $max$ is kept in a seperat pointer, so finding and accessing them is constant time. Since the arrays have a fixed length, independent from the size of the universe, \textbf{find\_pred}, \textbf{find\_succ} are also constant time.

\subsection*{Handling the recursive nodes}
The trees are build as described on the slides from the lectures - and \textbf{insert}, \textbf{delete} and \textbf{find\_succ} are all implemented as described on the slides as well. The operation \textbf{find\_pred}, is mostly identical to \textbf{find\_succ}, but obviously look for the first element preceeding the index. It's basicly a mirror of \textbf{find\_succ}.

\subsection*{Correctness}
Correctness have been loosely verified, by inserting 10 million random elements into a vEB tree, a binary heap and a Fibonacci heap - and then performing \textbf{delete\_min} until all the data structures are empty, while checking they all return the same value. This does not gurantee correctness, but the chance that all three implementations having the same behaviour is practically non-existing, due to the very different types of data structure. This is a good test since it's reasonable to assume that inserting 10 million random elements into the vEB tree will touch all insert cases, and same applies for delete (by \textbf{delete\_min}).

\subsection*{Futher improvements}
Out leafs are an ordinary ``nodes'' with an array instead of continuing the search. One possibility could be to use a leaf size of 32 or 64, depending og the machine (32 bit vs 64 bit), and represent the elements as a bit array. Manipulating this will be extremly efficint, and then keep pointers to the actual data somewhere else.